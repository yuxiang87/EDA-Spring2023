% Options for packages loaded elsewhere
\PassOptionsToPackage{unicode}{hyperref}
\PassOptionsToPackage{hyphens}{url}
%
\documentclass[
]{article}
\usepackage{amsmath,amssymb}
\usepackage{lmodern}
\usepackage{iftex}
\ifPDFTeX
  \usepackage[T1]{fontenc}
  \usepackage[utf8]{inputenc}
  \usepackage{textcomp} % provide euro and other symbols
\else % if luatex or xetex
  \usepackage{unicode-math}
  \defaultfontfeatures{Scale=MatchLowercase}
  \defaultfontfeatures[\rmfamily]{Ligatures=TeX,Scale=1}
\fi
% Use upquote if available, for straight quotes in verbatim environments
\IfFileExists{upquote.sty}{\usepackage{upquote}}{}
\IfFileExists{microtype.sty}{% use microtype if available
  \usepackage[]{microtype}
  \UseMicrotypeSet[protrusion]{basicmath} % disable protrusion for tt fonts
}{}
\makeatletter
\@ifundefined{KOMAClassName}{% if non-KOMA class
  \IfFileExists{parskip.sty}{%
    \usepackage{parskip}
  }{% else
    \setlength{\parindent}{0pt}
    \setlength{\parskip}{6pt plus 2pt minus 1pt}}
}{% if KOMA class
  \KOMAoptions{parskip=half}}
\makeatother
\usepackage{xcolor}
\usepackage[margin=2.54cm]{geometry}
\usepackage{graphicx}
\makeatletter
\def\maxwidth{\ifdim\Gin@nat@width>\linewidth\linewidth\else\Gin@nat@width\fi}
\def\maxheight{\ifdim\Gin@nat@height>\textheight\textheight\else\Gin@nat@height\fi}
\makeatother
% Scale images if necessary, so that they will not overflow the page
% margins by default, and it is still possible to overwrite the defaults
% using explicit options in \includegraphics[width, height, ...]{}
\setkeys{Gin}{width=\maxwidth,height=\maxheight,keepaspectratio}
% Set default figure placement to htbp
\makeatletter
\def\fps@figure{htbp}
\makeatother
\setlength{\emergencystretch}{3em} % prevent overfull lines
\providecommand{\tightlist}{%
  \setlength{\itemsep}{0pt}\setlength{\parskip}{0pt}}
\setcounter{secnumdepth}{-\maxdimen} % remove section numbering
\ifLuaTeX
  \usepackage{selnolig}  % disable illegal ligatures
\fi
\IfFileExists{bookmark.sty}{\usepackage{bookmark}}{\usepackage{hyperref}}
\IfFileExists{xurl.sty}{\usepackage{xurl}}{} % add URL line breaks if available
\urlstyle{same} % disable monospaced font for URLs
\hypersetup{
  pdftitle={Assignment 1: Introduction},
  pdfauthor={Yuxiang Ren},
  hidelinks,
  pdfcreator={LaTeX via pandoc}}

\title{Assignment 1: Introduction}
\author{Yuxiang Ren}
\date{}

\begin{document}
\maketitle

\hypertarget{overview}{%
\subsection{OVERVIEW}\label{overview}}

This exercise accompanies the introductory material in Environmental
Data Analytics.

\hypertarget{directions}{%
\subsection{Directions}\label{directions}}

\begin{enumerate}
\def\labelenumi{\arabic{enumi}.}
\tightlist
\item
  Rename this file
  \texttt{\textless{}FirstLast\textgreater{}\_A01\_Introduction.Rmd}
  (replacing \texttt{\textless{}FirstLast\textgreater{}} with your first
  and last name).
\item
  Change ``Student Name'' on line 3 (above) with your name.
\item
  Work through the steps, \textbf{creating code and output} that fulfill
  each instruction.
\item
  Be sure to \textbf{answer the questions} in this assignment document.
\item
  When you have completed the assignment, \textbf{Knit} the text and
  code into a single PDF file.
\item
  After Knitting, submit the completed exercise (PDF file) to the
  appropriate assigment section on Sakai.
\end{enumerate}

\hypertarget{finish-setting-up-r-studio}{%
\subsection{1) Finish setting up R
Studio}\label{finish-setting-up-r-studio}}

\hypertarget{install-tinytex}{%
\subsubsection{Install TinyTex}\label{install-tinytex}}

Now, run this code cell the same way. This will install ``tinytex'' -- a
helper app that allows you to knit your markdown documents into
professional quality PDFs.

\hypertarget{set-your-default-knit-directory}{%
\subsubsection{Set your default knit
directory}\label{set-your-default-knit-directory}}

This setting will help deal with relative paths later on\ldots{} - From
the Tool menu, select \texttt{Global\ Options} - Select the RMarkdown
section - In the ``Evaluate chunks in directory'', set the option to
``Project''

\hypertarget{discussion-questions}{%
\subsection{2) Discussion Questions}\label{discussion-questions}}

Enter answers to the questions just below the \textgreater Answer:
prompt.

\begin{enumerate}
\def\labelenumi{\arabic{enumi}.}
\tightlist
\item
  What are your previous experiences with data analytics, R, and Git?
  Include both formal and informal training.
\end{enumerate}

\begin{quote}
Answer: I am a student from the international Master of Environmental
Policy (iMEP) program, and I used R through course PUBPOL870K last
spring. We learned some statistics knowledge through R, such as
frequency, sampling, regression, etc.
\end{quote}

\begin{enumerate}
\def\labelenumi{\arabic{enumi}.}
\setcounter{enumi}{1}
\tightlist
\item
  Are there any components of the course about which you feel confident?
\end{enumerate}

\begin{quote}
Answer: I know some statistical logic that might help me understand the
R coding process.
\end{quote}

\begin{enumerate}
\def\labelenumi{\arabic{enumi}.}
\setcounter{enumi}{2}
\tightlist
\item
  Are there any components of the course about which you feel
  apprehensive?
\end{enumerate}

\begin{quote}
Answer: The previous bad experience with R was caused by the operating
system is Mac. After reading and following the web guides, i have
successfully installed all the necessary soft ware, so this might no
longer be a problem.
\end{quote}

\hypertarget{github}{%
\subsection{3) GitHub}\label{github}}

Provide a link below to your forked course repository in GitHub. Make
sure you have pulled all recent changes from the course repository and
that you have updated your course README file, committed those changes,
and pushed them to your GitHub account.

\begin{quote}
Answer:\url{https://github.com/yuxiang87/EDA-Spring2023.git}
\end{quote}

\hypertarget{knitting}{%
\subsection{4) Knitting}\label{knitting}}

When you have completed this document, click the \texttt{knit}
button.This should produce a PDF copy of your markdown document. Submit
this PDF to Sakai.

\end{document}
